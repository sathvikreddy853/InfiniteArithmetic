
\sethlcolor{lightgray}  %change colour here
\colorlet{Gray}{gray!25}
\tcbset{on line, 
        boxsep=1.5pt, left=0pt,right=0pt,top=0pt,bottom=0pt,
        colframe=white,colback=Gray,  
        highlight math style={enhanced}
        }

\section{API Tables}
Note: the ones prefixed with @ are private methods.
\subsection{Integer Table}
\begin{table}[h]
  \centering
  \renewcommand{\arraystretch}{1.5} % Adjust the height of each row
  \begin{tabular}{ >{\raggedright\arraybackslash}p{4cm}>{\raggedright\arraybackslash}p{8cm}} % Defines column widths and borders
    \hline
    \thead{\large\textbf{Functions}} & \thead{\large\textbf{Purpose}} \\
    \hline
    \tcbox{\textbf{Assign}} & Used to assign the object to another value \\
    \hline
    \tcbox{\textbf{Add}} & Adds two numbers together \\
    \hline
    \tcbox{\textbf{Subtract}} & Subtracts one number from the other\\
    \hline
    \tcbox{\textbf{Multiply}} & Multiplies two numbers \\
    \hline
    \tcbox{\textbf{Divide}} & Divides two numbers \\
    \hline
    \tcbox{\textbf{Negate}} & Negates the number \\
    \hline
    \tcbox{\textbf{isZero}} & Tells if the number is zero or not \\
    \hline
    \tcbox{\textbf{parse}} & Returns an instance of the Integer class \\
	\hline
	\tcbox{\textbf{Compare}} & Compares the object to another and returns a value among \verb|{0, 1, 2}| \\
	\hline
	\tcbox{\textbf{Negate}} & Negates the object \\
    \hline
    \tcbox{\textbf{@Complement}} & Takes the complement of given number \\
    \hline
    \tcbox{\textbf{@MatchDigits}} & Matches the digits of two numbers such that they are equal \\
    \hline
    \tcbox{\textbf{@PopZero}} & Removes redundant zeroes from the number \\
    \hline
    \tcbox{\textbf{@Print}} & Displays the number \\
    \hline
  \end{tabular}

  \centering
  \renewcommand{\arraystretch}{1.5} % Adjust the height of each row
  \begin{tabular}{ >{\raggedright\arraybackslash}p{4cm}>{\raggedright\arraybackslash}p{8cm}} % Defines column widths and borders
    \hline
    \thead{\large\textbf{Operators}} & \thead{\large\textbf{Purpose}} \\
    \hline
    \tcbox{\textbf{operator\tiny{\textless\textless}}} & Used to assign the object to another value \\
    \hline
    \tcbox{\textbf{operator\tiny{\textgreater\textgreater}}} & Used to take the number from input stream \\
    \hline
    \tcbox{\textbf{operator=}} & Used to assign the object to another value\\
    \hline
    \tcbox{\textbf{operator+}} & Adds two numbers \\
    \hline
    \tcbox{\textbf{operator-}} & Subtracts one number from the other \\
    \hline
    \tcbox{\textbf{operator*}} & Multiplies two numbers \\
    \hline
    \tcbox{\textbf{operator/}} & Divides one number by the other \\
    \hline
    \tcbox{\textbf{operator\%}} & Takes the mod of one number w.r.t the other \\
    \hline
    \tcbox{\textbf{operator+}} & Returns the same number \\
    \hline
    \tcbox{\textbf{operator-}} & Negates the number \\
    \hline
    \tcbox{\textbf{operator\~}} & Takes the complement of the number \\
    \hline
  \end{tabular}
\end{table}
\newpage
\subsection{Float Table}
\begin{table}[h]
  \centering
  \renewcommand{\arraystretch}{1.5} % Adjust the height of each row
  \begin{tabular}{ >{\raggedright\arraybackslash}p{4cm}>{\raggedright\arraybackslash}p{8cm}} % Defines column widths and borders
    \hline
    \thead{\large\textbf{Functions}} & \thead{\large\textbf{Purpose}} \\
    \hline
    \tcbox{\textbf{Assign}} & Used to assign the object to another value \\
    \hline
    \tcbox{\textbf{Add}} & Adds two numbers together \\
    \hline
    \tcbox{\textbf{Subtract}} & Subtracts one number from the other\\
    \hline
    \tcbox{\textbf{Multiply}} & Multiplies two numbers \\
    \hline
    \tcbox{\textbf{Divide}} & Divides two numbers \\
    \hline
    \tcbox{\textbf{Negate}} & Negates the number \\
    \hline
    \tcbox{\textbf{isZero}} & Tells if the number is zero or not \\
    \hline
    \tcbox{\textbf{parse}} & Returns an instance of the Float class \\
	\hline
	\tcbox{\textbf{Compare}} & Compares the object to another and returns a value among \verb|{0, 1, 2}| \\
	\hline
	\tcbox{\textbf{Negate}} & Negates the object \\
    \hline
    \tcbox{\textbf{@Complement}} & Takes the complement of given number \\
    \hline
    \tcbox{\textbf{@MatchDigits}} & Matches the digits of two numbers such that they are equal \\
    \hline
    \tcbox{\textbf{@PopZero}} & Removes redundant zeroes from the number \\
    \hline
    \tcbox{\textbf{@Print}} & Displays the number \\
    \hline
  \end{tabular}
  \centering
  \renewcommand{\arraystretch}{1.5} % Adjust the height of each row
  \begin{tabular}{ >{\raggedright\arraybackslash}p{4cm}>{\raggedright\arraybackslash}p{8cm}} % Defines column widths and borders
    \hline
    \thead{\large\textbf{Operators}} & \thead{\large\textbf{Purpose}} \\
    \hline
    \tcbox{\textbf{operator\tiny{\textless\textless}}} & Used to assign the object to another value \\
    \hline
    \tcbox{\textbf{operator\tiny{\textgreater\textgreater}}} & Used to take the number from input stream \\
    \hline
    \tcbox{\textbf{operator=}} & Used to assign the object to another value\\
    \hline
    \tcbox{\textbf{operator+}} & Adds two numbers \\
    \hline
    \tcbox{\textbf{operator-}} & Subtracts one number from the other \\
    \hline
    \tcbox{\textbf{operator*}} & Multiplies two numbers \\
    \hline
    \tcbox{\textbf{operator/}} & Divides one number by the other \\
    \hline
    \tcbox{\textbf{operator+}} & Returns the same number \\
    \hline
    \tcbox{\textbf{operator-}} & Negates the number \\
    \hline
    \tcbox{\textbf{operator\~}} & Takes the complement of the number \\
    \hline
  \end{tabular}
\end{table}

