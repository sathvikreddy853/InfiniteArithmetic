\section{Float Library}

\subsection{Introduction}

Describe the purpose and functionalities of your Float library here. Explain how it provides infinite precision arithmetic for floating-point numbers. 

\subsection{API Reference}

Document the functions and methods provided by the Float library, similar to the Integer library section.
\subsubsection{Constructors}
\subsubsection{Destructor}
\subsubsection{parse}
\subsubsection{Assign}
\subsubsection{Print}
\subsubsection{Input}
\subsubsection{Add}
\subsubsection{Subtract}
\subsubsection{Multiply}
\subsubsection{MultiplyByDigit}
\subsubsection{Divide}
\subsubsection{SetPrecision}
\subsubsection{Compare}
\subsubsection{Complement}
\subsubsection{Negate}
\subsubsection{isZero}
\subsubsection{MatchDigits}
\subsubsection{ResizeEnds}
\subsubsection{VerifyString}
\subsubsection{PopZero}

\begin{itemize}
    \item \textbf{add(a, b)}: Adds two floats `a` and `b` and returns the result as a Float object.
    \item \textbf{subtract(a, b)}: Subtracts float `b` from `a` and returns the result as a Float object.
    \item \textbf{multiply(a, b)}: Multiplies two floats `a` and `b` and returns the result as a Float object.
    \item \textbf{divide(a, b)}: Divides float `a` by `b` and returns the result as a Float object. (Handle division by zero case)
    \item \textbf{abs(a)}: Returns the absolute value of float `a` as a Float object.
    \item \textbf{toString()}: Converts the Float object to a string representation.
    % Add other relevant functions with explanations.
\end{itemize}

\subsection{Examples}

Include code examples demonstrating the usage of the Float library functions. 

\begin{verbatim}
>>> a = Float(3.14)
>>> b = Float(2.72)
>>> c = a.add(b)
>>> print(c.toString())  # Output: 5.86

>>> d = a.divide(b)
>>> print(d.toString())  # Output: 1.15... (show limited precision)
\end{verbatim}

