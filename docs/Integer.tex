% Integer Library

\section{Integer Library}

\subsection{Introduction}

In this section, describe the purpose and functionalities of your Integer library. Explain how it provides infinite precision arithmetic for integers. 

\subsection{API Reference}

Here, list and explain the functions and methods provided by the Integer library. Use clear and concise explanations.

\subsubsection{Constructors} \vspace*{1em}

{\ttfamily \large Integer()} \\[2mm]
This constructor creates a vector initializes it to \verb|{0}| and initializes \tt{isNegative} to \tt{false}.
\vspace*{1em}
\begin{lstlisting}[language = C]
#include "Integer.h"
#include "Float.h"

int main()
{
	LOG(InfiniteArithmetic::Integer());
	// Output = 0
}
\end{lstlisting}
\vspace*{1em}


\noindent
{\ttfamily \large Integer(std::string num)} \\[2mm]
This constructor sets the member variables to appropriate values based on the string. It calls the \verb|VerifyString| function internally to check if the string provided is valid or not.
\vspace*{1em}
\begin{lstlisting}[language = C]

	LOG(InfiniteArithmetic::Integer("212"));
	// Output = 212
.
\end{lstlisting}
\vspace*{1em}


\noindent
{\ttfamily \large Integer(const Integer \&obj)} \\[2mm]
It is a copy constructor that replicates the values of the object given.
\vspace*{1em}
\begin{lstlisting}[language = C]

	namespace InfiniteArithmetic = IA;
	LOG(IA::Integer(IA::Integer("110")));
	// Output = 110
.
\end{lstlisting}
\vspace*{1em}


\subsubsection{Destructor}
{\ttfamily \large \~{}Integer()} \\[2mm]
The destructor deletes the vector and the boolean variable \verb|isNegative| explicitly.


\subsubsection{Integer parse(const std::string \&)}
The parse function returns an instance of the \verb|Integer| class.


\subsubsection{void Assign(Integer)}
The \verb|Assign| function assigns the value of one integer to the other.
\vspace*{1em}
\begin{lstlisting}[language = C]	

	IA::Integer num1 ("102");
	IA::Integer num2;
	num2.Assign(num1);
	LOG(num2);
	// Output = 102
.
\end{lstlisting}
\vspace*{1em}

The \verb|= operator| is an other function that has been overloaded to assign one variable to another.


\subsubsection{Print}
The \verb|Print| is a private function that is used to display the contents of the vector. \\

However, the \verb|<<| (insertion operator) has been overloaded.
\subsubsection{Input}
Input can be taken through \verb|>>| (extraction operator). 
\subsubsection{Add}
\subsubsection{Add2}
\subsubsection{Subtract}
\subsubsection{Multiply}
\subsubsection{MultiplyByDigit}
\subsubsection{Divide}
\subsubsection{Mod}
\subsubsection{Compare}
\subsubsection{Complement}
\subsubsection{Negate}
\subsubsection{isZero}
\subsubsection{MatchDigits}
\subsubsection{VerifyString}
\subsubsection{PopZero}

% \begin{itemize}
%     \item \textbf{add(a, b)}: Adds two integers `a` and `b` and returns the result as an Integer object.
%     \item \textbf{subtract(a, b)}: Subtracts integer `b` from `a` and returns the result as an Integer object.
%     \item \textbf{multiply(a, b)}: Multiplies two integers `a` and `b` and returns the result as an Integer object.
%     \item \textbf{divide(a, b)}: Divides integer `a` by `b` and returns the result as an Integer object. (Handle division by zero case)
%     \item \textbf{abs(a)}: Returns the absolute value of integer `a` as an Integer object.
%     \item \textbf{toString()}: Converts the Integer object to a string representation.
%     % Add other relevant functions with explanations.
% \end{itemize}

\subsection{Examples}

Provide code examples demonstrating the usage of the Integer library functions. 

\begin{verbatim}
>>> a = Integer(10)
>>> b = Integer(5)
>>> c = a.add(b)
>>> print(c.toString())  # Output: 15

>>> d = a.divide(b)
>>> print(d.toString())  # Output: 2
\end{verbatim}