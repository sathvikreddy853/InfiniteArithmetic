% Integer Library

\section{Integer Library}

\subsection{Introduction}
Given below is an API reference for Integer:

\subsection{API Reference}

Here is a list of a functions that are a part of the \verb|Integer| class. 

\subsubsection{Constructors} \vspace*{1em}

{\ttfamily \large Integer()} \\[2mm]
This constructor creates a vector initializes it to \verb|{0}| and initializes \tt{isNegative} to \tt{false}.
\vspace*{1em}
\begin{lstlisting}[language = C]
#include "Integer.h"
#include "Float.h"

int main()
{
	LOG(InfiniteArithmetic::Integer());
	// Output = 0
}
\end{lstlisting}
\vspace*{1em}


\noindent
{\ttfamily \large Integer(std::string num)} \\[2mm]
This constructor sets the member variables to appropriate values based on the string. It calls the \verb|VerifyString| function internally to check if the string provided is valid or not.
\vspace*{1em}
\begin{lstlisting}[language = C]

	LOG(InfiniteArithmetic::Integer("212"));
	// Output = 212
.
\end{lstlisting}
\vspace*{1em}


\noindent
{\ttfamily \large Integer(const Integer \&obj)} \\[2mm]
It is a copy constructor that replicates the values of the object given.
\vspace*{1em}
\begin{lstlisting}[language = C]

	namespace InfiniteArithmetic = IA;
	LOG(IA::Integer(IA::Integer("110")));
	// Output = 110
.
\end{lstlisting}
\vspace*{1em}


\subsubsection{Destructor}  \vspace*{0.5em}

{\ttfamily \large \~{}Integer()} \\[2mm]
The destructor deletes the vector and the boolean variable \verb|isNegative| explicitly.


\subsubsection{parse} 

{\ttfamily \large Integer parse(const std::string \&)} \\[2mm]
The parse function returns an instance of the \verb|Integer| class.

\subsubsection{Assign}  
{\ttfamily \large void Assign(Integer)} \\[2mm]
The \verb|Assign| function assigns the value of one integer to the other.
\vspace*{1em}
\begin{lstlisting}[language = C]	

	IA::Integer num1 ("102");
	IA::Integer num2;
	num2.Assign(num1);
	LOG(num2);
	// Output = 102
.
\end{lstlisting}
\vspace*{1em}

The \verb|= operator| is an other function that has been overloaded to assign one variable to another.


\subsubsection{Print}
The \verb|Print| is a private function that is used to display the contents of the vector. \\
However, the \verb|<<| operator (insertion operator) has been overloaded.
\vspace*{1em}
\begin{lstlisting}[language = C]

	using namespace InfiniteArithmetic;

	Integer num1("322");

	std::cout << num1  << std::endl; 
	// Output = 322
.
\end{lstlisting}
\vspace*{1em}

\subsubsection{Input}
Input can be taken through \verb|>>| operator (extraction operator). 
\vspace*{1em}
\begin{lstlisting}[language = C]	

	using namespace InfiniteArithmetic;

	Integer num1;
	Integer num2;

	std::cin >> num1 >> num2;
.
\end{lstlisting}
\vspace*{1em}

\subsubsection{Add}
{\ttfamily \large Integer Add(Integer)} \\[2mm]
The \verb|Add| function takes in two copies of \verb|Integer| as it modifies them internally. It call the \verb|MatchDigits| function to adjust the size of the inputs (for proper addition). \\
The \verb|+| operator has been overloaded. It internally calls the \verb|Add| function.
\vspace*{1em}
\begin{lstlisting}[language = C]	

	using namespace InfiniteArithmetic;

	Integer num1 ("1023");
	Integer num2 ("3213");

	LOG(num1.Add(num2));
	// Output = 4236
.
\end{lstlisting}
\vspace*{1em}

\subsubsection{Add2}
{\ttfamily \large Integer Add2(Integer \&)} \\[2mm]
The \verb|Add2| function has a small modification in that it is a little more effiecient(does $3$ seconds better over $1000000$ testcases). It takes in the variables as references and does not call the \verb|MatchDigits| function.
\vspace*{1em}
\begin{lstlisting}[language = C]	
	
	using namespace InfiniteArithmetic;

	Integer num1 ("1023");
	Integer num2 ("3213");

	LOG(num1.Add2(num2));
	// Output = 4236
.
\end{lstlisting}
\vspace*{1em}


\subsubsection{Subtract}
{\ttfamily \large Integer Subtract(Integer)} \\[2mm]
The \verb|Subtract| function negates the second number and calls the \verb|Add| function. \\
The \verb|-| operator has been overloaded. It internally calls the \verb|Subtract| function.
\vspace*{1em}
\begin{lstlisting}[language = C]	
	using namespace InfiniteArithmetic;

	Integer num1 ("1023");
	Integer num2 ("3213");

	LOG(num1.Subtract(num2));
	// Output = -2190
.
\end{lstlisting}
\vspace*{1em}


\subsubsection{Multiply}
{\ttfamily \large Integer Multiply(Integer)} \\[2mm]
The \verb|*| operator has been overloaded for multiplication.
\vspace*{1em}
\begin{lstlisting}[language = C]	
	using namespace InfiniteArithmetic;

	Integer num1 ("322");
	Integer num2 ("4221");

	LOG(num1.Multiply(num2));
	// Output = 1359162
.
\end{lstlisting}
\vspace*{1em}


\subsubsection{MultiplyByDigit}
{\ttfamily \large Integer MultiplyByDigit(int)} \\[2mm]
This function is called internally by \verb|Divide|. It multiplies an \verb|Integer| by a single digit\verb|(0-9)|.
\vspace*{0em}
\begin{lstlisting}[language = C]

	using namespace InfiniteArithmetic;

	Integer num1 ("322");
	LOG(num1.MultiplyByDigit(2));
	// Output = 644
.
\end{lstlisting}


\subsubsection{Divide}
{\ttfamily \large Integer Divide(Integer)} \\[2mm]
The function \verb|Divide| internally does human long division. The \verb|/| operator has been overloaded for division. Here's an example:
\vspace*{1em}
\begin{lstlisting}[language = C]	

	using namespace InfiniteArithmetic;

	Integer num1 ("322");
	Integer num2 ("22");

	LOG(num1.Divide(num2));
	// Output = 14
.
\end{lstlisting}
\vspace*{1em}


\subsubsection{Mod}
{\ttfamily \large Integer Mod(Integer)} \\[2mm]
The \verb|Mod| function returns the remainder after a division. It is calculated using the formula $p - (p/q)*q$ where $p$ and $q$ represent \verb|Integer| objects.
\vspace*{1em}
\begin{lstlisting}[language = C]

	using namespace InfiniteArithmetic;

	Integer num1 ("322");
	Integer num2 ("23");

	LOG(num1.Mod(num2));
	// Output = 14
.
\end{lstlisting}
\vspace*{1em}


\subsubsection{Compare}
{\ttfamily \large int16\_t Compare(Integer)} \\[2mm]
The \verb|Compare| function returns a value in \verb|{-1, 0, 1}| depending on which number is greater or lesser. All the comparison operators have been overloaded to do the same.
\vspace*{1em}
\begin{lstlisting}[language = C]

	using namespace InfiniteArithmetic;

	Integer num1 ("2");
	Integer num2 ("5");

	LOG(num1.Compare(num2));
	// Output = -1
.
\end{lstlisting}
\vspace*{1em}


\subsubsection{Complement}
{\ttfamily \large Integer Complement() const} \\[2mm]
The \verb|Complement| function takes the complement of a number with respect to 9. For example, $102$'s complement is $898$. \\
It is a private member function.


\subsubsection{Negate}
{\ttfamily \large Integer Negate()} \\[2mm]
The \verb|Negate| function changes the sign of the number. The unary \verb|-| operator also does the same.
\vspace*{1em}
\begin{lstlisting}[language = C]
	
	using namespace InfiniteArithmetic;

	Integer num1 ("5");
	LOG(num1.Negate());
	// Output = -5
.
\end{lstlisting}
\vspace*{1em}


\subsubsection{isZero}
{\ttfamily \large bool isZero()} \\[2mm]
The \verb|isZero| function checks if the number is $0$.
\vspace*{1em}
\begin{lstlisting}[language = C]

	using namespace InfiniteArithmetic;

	Integer num1 ("0");
	LOG(num1.isZero());
	// Output = 1

	Integer num2 ("2");
	LOG(num2.isZero());
	// Output = 0
.
\end{lstlisting}
\vspace*{1em}


\subsubsection{MatchDigits}
{\ttfamily \large void MatchDigits(Integer \&, Integer \&)} \\[2mm]
The \verb|MatchDigits| is a static function that makes the vectors of the two numbers to equal length. This is a private function.

\subsubsection{VerifyString}
{\ttfamily \large void VerifyString(std::string)} \\[2mm]
The \verb|VerifyString| checks for any non-digit characters and raises an error if found. This is also a private function.

\subsubsection{PopZero}
{\ttfamily \large void PopZero()} \\[2mm]
This function removes any redundant zeroes present in the number.

% \begin{itemize}
%     \item \textbf{add(a, b)}: Adds two integers `a` and `b` and returns the result as an Integer object.
%     \item \textbf{subtract(a, b)}: Subtracts integer `b` from `a` and returns the result as an Integer object.
%     \item \textbf{multiply(a, b)}: Multiplies two integers `a` and `b` and returns the result as an Integer object.
%     \item \textbf{divide(a, b)}: Divides integer `a` by `b` and returns the result as an Integer object. (Handle division by zero case)
%     \item \textbf{abs(a)}: Returns the absolute value of integer `a` as an Integer object.
%     \item \textbf{toString()}: Converts the Integer object to a string representation.
%     % Add other relevant functions with explanations.
% \end{itemize}